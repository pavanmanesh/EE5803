\documentclass[journal,12pt]{IEEEtran}
\usepackage{circuitikz}
\usepackage{tikz}
\usetikzlibrary{shapes.gates.logic.US}
\usetikzlibrary{circuits.ee.IEC}
\usepackage{longtable}
\usepackage{setspace}
\usepackage{gensymb}
\singlespacing
\usepackage[cmex10]{amsmath}
\newcommand\myemptypage{
	\null
	\thispagestyle{empty}
	\addtocounter{page}{-1}
	\newpage
}
\usepackage{amsthm}
\usepackage{mdframed}
\usepackage{mathrsfs}
\usepackage{txfonts}
\usepackage{stfloats}
\usepackage{bm}
\usepackage{cite}
\usepackage{cases}
\usepackage{subfig}

\usepackage{longtable}
\usepackage{multirow}

\usepackage{enumitem}
\usepackage{mathtools}
\usepackage{steinmetz}
\usepackage{verbatim}
\usepackage{tfrupee}
\usepackage[breaklinks=true]{hyperref}
\usepackage{graphicx}
\usepackage{tkz-euclide}
\usetikzlibrary{karnaugh}
\usetikzlibrary{calc,math}
\usepackage{listings}
\usepackage{karnaugh-map}
    \usepackage{color}                                            %%
    \usepackage{array}                                            %%
    \usepackage{longtable}                                        %%
    \usepackage{calc}                                             %%
    \usepackage{multirow}                                         %%
    \usepackage{hhline}                                           %%
    \usepackage{ifthen}                                           %%
    \usepackage{lscape}     
\usepackage{multicol}
\usepackage{chngcntr}

\DeclareMathOperator*{\Res}{Res}

\renewcommand\thesection{\arabic{section}}
\renewcommand\thesubsection{\thesection.\arabic{subsection}}
\renewcommand\thesubsubsection{\thesubsection.\arabic{subsubsection}}

\renewcommand\thesectiondis{\arabic{section}}
\renewcommand\thesubsectiondis{\thesectiondis.\arabic{subsection}}
\renewcommand\thesubsubsectiondis{\thesubsectiondis.\arabic{subsubsection}}


\hyphenation{op-tical net-works semi-conduc-tor}
\def\inputGnumericTable{}                                 %%

\lstset{
%language=C,
frame=single, 
breaklines=true,
columns=fullflexible
}
\begin{document}
\onecolumn

\newtheorem{theorem}{Theorem}[section]
\newtheorem{problem}{Problem}
\newtheorem{proposition}{Proposition}[section]
\newtheorem{lemma}{Lemma}[section]
\newtheorem{corollary}[theorem]{Corollary}
\newtheorem{example}{Example}[section]
\newtheorem{definition}[problem]{Definition}

\newcommand{\BEQA}{\begin{eqnarray}}
\newcommand{\EEQA}{\end{eqnarray}}
\newcommand{\define}{\stackrel{\triangle}{=}}
\bibliographystyle{IEEEtran}
\raggedbottom
\setlength{\parindent}{0pt}
\providecommand{\mbf}{\mathbf}
\providecommand{\pr}[1]{\ensuremath{\Pr\left(#1\right)}}
\providecommand{\qfunc}[1]{\ensuremath{Q\left(#1\right)}}
\providecommand{\sbrak}[1]{\ensuremath{{}\left[#1\right]}}
\providecommand{\lsbrak}[1]{\ensuremath{{}\left[#1\right.}}
\providecommand{\rsbrak}[1]{\ensuremath{{}\left.#1\right]}}
\providecommand{\brak}[1]{\ensuremath{\left(#1\right)}}
\providecommand{\lbrak}[1]{\ensuremath{\left(#1\right.}}
\providecommand{\rbrak}[1]{\ensuremath{\left.#1\right)}}
\providecommand{\cbrak}[1]{\ensuremath{\left\{#1\right\}}}
\providecommand{\lcbrak}[1]{\ensuremath{\left\{#1\right.}}
\providecommand{\rcbrak}[1]{\ensuremath{\left.#1\right\}}}
\theoremstyle{remark}
\newtheorem{rem}{Remark}
\providecommand{\dec}[2]{\ensuremath{\overset{#1}{\underset{#2}{\gtrless}}}}
\newcommand{\myvec}[1]{\ensuremath{\begin{pmatrix}#1\end{pmatrix}}}
\renewcommand{\thefigure}{\theproblem}
\def\putbox#1#2#3{\makebox[0in][l]{\makebox[#1][l]{}\raisebox{\baselineskip}[0in][0in]{\raisebox{#2}[0in][0in]{#3}}}}
     \def\rightbox#1{\makebox[0in][r]{#1}}
     \def\centbox#1{\makebox[0in]{#1}}
     \def\topbox#1{\raisebox{-\baselineskip}[0in][0in]{#1}}
     \def\midbox#1{\raisebox{-0.5\baselineskip}[0in][0in]{#1}}
\vspace{3cm}
\title{Assignment-1}
\author{M Pavan Manesh - EE20MTECH14017}
\maketitle
\bigskip
\renewcommand{\thefigure}{\theenumi}
\renewcommand{\thetable}{\theenumi}
%
Download the codes from 
%
\begin{lstlisting}
https://github.com/pavanmanesh/EE5803/tree/main/assign1
\end{lstlisting}
\section{\textbf{Problem}}
Verify the following using Boolean Laws.
\begin{align}
    A^{'}+B^{'}.C=A^{'}.B^{'}.C^{'}+A^{'}.B.C^{'}+A^{'}.B.C+A^{'}.B^{'}.C+A.B^{'}.C \nonumber
\end{align}
\section{\textbf{Solution}}
From the Figure \ref{fig:kmap},We can see that the right hand side expression is simplified to that of left side expression.
\begin{figure}[h!]
    \centering
    \begin{karnaugh-map}[4][2][1][][]
        
        \minterms{0,1,2,3,5}
        \maxterms{4,6,7}
        \implicant{0}{2}
        \implicant{1}{5}
        \draw[color=black, ultra thin] (0, 2) --
        node [pos=0.7, above right, anchor=south west] {$BC$} % Y label
        node [pos=0.7, below left, anchor=north east] {$A$} % X label
        ++(135:1);
    \end{karnaugh-map}
    \caption{K-Map}
    \label{fig:kmap}
\end{figure}
\section{\textbf{Truth Table}}
The truth table corresponding to the given expression:
\begin{table}[h]
    \centering
    \begin{tabular}{|c|c|c|c|c|}
    \hline
    $A$&$B$&$C$&LHS&RHS  \\
    \hline
    0&0&0&1&1\\
    0&0&1&1&1\\
    0&1&0&1&1\\
    0&1&1&1&1\\
    1&0&0&0&0\\
    1&0&1&1&1\\
    1&1&0&0&0\\
    1&1&1&0&0\\
    \hline
    \end{tabular}
    \caption{Truth Table}
    \label{table}
\end{table}
\section{\textbf{Implementation Using NAND Logic}}
\begin{align}
    \overline{A}+\overline{B}.C=\overline{\overline{\overline{A}+\overline{B}.C}}=\overline{\overline{\overline{A}}.(\overline{\overline{B}.C})}=\overline{A.(\overline{\overline{B}.C})}
\end{align}
\newpage
\begin{figure}[h!]
\centering
\begin{tikzpicture}[]
    \node (A) at (0,4.5) {$A$};
    \node (B) at (0,2.7) {$B$};
    \node (C) at (0,0.7) {$C$};
    \node[nand gate US, minimum size=32pt, draw, logic gate inputs=nn] at ($(A) + (7.5, -0.2)$) (nand1) {};
    \node[nand gate US, minimum size=32pt, draw, logic gate inputs=nn] at ($(A) + (2, -1.8)$) (nand2) {};
     \node[nand gate US, minimum size=32pt, draw, logic gate inputs=nn] at ($(A) + (4.5, -3.6)$) (nand3) {};
     
    \draw (B.east) - ++(right:5mm) |- (nand2.input 1);
    \draw (B.east) - ++(right:5mm) |- (nand2.input 2);
    \draw (A) -- ([xshift=0.2cm]A) |- (nand1.input 1);

    \draw (nand2.output) -- node[above]{$\overline{B}$} ([xshift=0.2cm]nand2.output) |- (nand3.input 1);
    \draw (C) -- ([xshift=0.2cm]C) |- (nand3.input 2);
    \draw (nand3.output) -- node[above]{$\overline{\overline{B}.C}$}([xshift=0.9cm]nand3.output) |- (nand1.input 2);
    \draw (nand1.output) -- node[above]{$\overline{A.(\overline{\overline{B}.C})}$} ($(nand1) + (2, 0)$);
    \end{tikzpicture}
    \caption{Logic circuit using NAND logic}
    \label{fig:NAND logic ckt}
\end{figure}
\end{document}
