\documentclass[journal,12pt]{IEEEtran}
\usepackage{longtable}
\usepackage{setspace}
\usepackage{gensymb}
\singlespacing
\usepackage[cmex10]{amsmath}
\newcommand\myemptypage{
	\null
	\thispagestyle{empty}
	\addtocounter{page}{-1}
	\newpage
}
\usepackage{amsthm}
\usepackage{mdframed}
\usepackage{mathrsfs}
\usepackage{txfonts}
\usepackage{stfloats}
\usepackage{bm}
\usepackage{cite}
\usepackage{cases}
\usepackage{subfig}

\usepackage{longtable}
\usepackage{multirow}

\usepackage{enumitem}
\usepackage{mathtools}
\usepackage{steinmetz}
\usepackage{tikz}
\usepackage{circuitikz}
\usepackage{verbatim}
\usepackage{tfrupee}
\usepackage[breaklinks=true]{hyperref}
\usepackage{graphicx}
\usepackage{tkz-euclide}

\usetikzlibrary{calc,math}
\usepackage{listings}
    \usepackage{color}                                            %%
    \usepackage{array}                                            %%
    \usepackage{longtable}                                        %%
    \usepackage{calc}                                             %%
    \usepackage{multirow}                                         %%
    \usepackage{hhline}                                           %%
    \usepackage{ifthen}                                           %%
    \usepackage{lscape}     
\usepackage{multicol}
\usepackage{chngcntr}

\DeclareMathOperator*{\Res}{Res}

\renewcommand\thesection{\arabic{section}}
\renewcommand\thesubsection{\thesection.\arabic{subsection}}
\renewcommand\thesubsubsection{\thesubsection.\arabic{subsubsection}}

\renewcommand\thesectiondis{\arabic{section}}
\renewcommand\thesubsectiondis{\thesectiondis.\arabic{subsection}}
\renewcommand\thesubsubsectiondis{\thesubsectiondis.\arabic{subsubsection}}


\hyphenation{op-tical net-works semi-conduc-tor}
\def\inputGnumericTable{}                                 %%

\lstset{
%language=C,
frame=single, 
breaklines=true,
columns=fullflexible
}
\begin{document}
\onecolumn

\newtheorem{theorem}{Theorem}[section]
\newtheorem{problem}{Problem}
\newtheorem{proposition}{Proposition}[section]
\newtheorem{lemma}{Lemma}[section]
\newtheorem{corollary}[theorem]{Corollary}
\newtheorem{example}{Example}[section]
\newtheorem{definition}[problem]{Definition}

\newcommand{\BEQA}{\begin{eqnarray}}
\newcommand{\EEQA}{\end{eqnarray}}
\newcommand{\define}{\stackrel{\triangle}{=}}
\bibliographystyle{IEEEtran}
\raggedbottom
\setlength{\parindent}{0pt}
\providecommand{\mbf}{\mathbf}
\providecommand{\pr}[1]{\ensuremath{\Pr\left(#1\right)}}
\providecommand{\qfunc}[1]{\ensuremath{Q\left(#1\right)}}
\providecommand{\sbrak}[1]{\ensuremath{{}\left[#1\right]}}
\providecommand{\lsbrak}[1]{\ensuremath{{}\left[#1\right.}}
\providecommand{\rsbrak}[1]{\ensuremath{{}\left.#1\right]}}
\providecommand{\brak}[1]{\ensuremath{\left(#1\right)}}
\providecommand{\lbrak}[1]{\ensuremath{\left(#1\right.}}
\providecommand{\rbrak}[1]{\ensuremath{\left.#1\right)}}
\providecommand{\cbrak}[1]{\ensuremath{\left\{#1\right\}}}
\providecommand{\lcbrak}[1]{\ensuremath{\left\{#1\right.}}
\providecommand{\rcbrak}[1]{\ensuremath{\left.#1\right\}}}
\theoremstyle{remark}
\newtheorem{rem}{Remark}
\newcommand{\sgn}{\mathop{\mathrm{sgn}}}
\providecommand{\abs}[1]{\left\vert#1\right\vert}
\providecommand{\res}[1]{\Res\displaylimits_{#1}} 
\providecommand{\norm}[1]{\left\lVert#1\right\rVert}
%\providecommand{\norm}[1]{\lVert#1\rVert}
\providecommand{\mtx}[1]{\mathbf{#1}}
\providecommand{\mean}[1]{E\left[ #1 \right]}
\providecommand{\fourier}{\overset{\mathcal{F}}{ \rightleftharpoons}}
%\providecommand{\hilbert}{\overset{\mathcal{H}}{ \rightleftharpoons}}
\providecommand{\system}{\overset{\mathcal{H}}{ \longleftrightarrow}}
	%\newcommand{\solution}[2]{\textbf{Solution:}{#1}}
\newcommand{\solution}{\noindent \textbf{Solution: }}
\newcommand{\cosec}{\,\text{cosec}\,}
\providecommand{\dec}[2]{\ensuremath{\overset{#1}{\underset{#2}{\gtrless}}}}
\newcommand{\myvec}[1]{\ensuremath{\begin{pmatrix}#1\end{pmatrix}}}
\newcommand{\mydet}[1]{\ensuremath{\begin{vmatrix}#1\end{vmatrix}}}
\numberwithin{equation}{subsection}
\makeatletter
\@addtoreset{figure}{problem}
\makeatother
\let\StandardTheFigure\thefigure
\let\vec\mathbf
\renewcommand{\thefigure}{\theproblem}
\def\putbox#1#2#3{\makebox[0in][l]{\makebox[#1][l]{}\raisebox{\baselineskip}[0in][0in]{\raisebox{#2}[0in][0in]{#3}}}}
     \def\rightbox#1{\makebox[0in][r]{#1}}
     \def\centbox#1{\makebox[0in]{#1}}
     \def\topbox#1{\raisebox{-\baselineskip}[0in][0in]{#1}}
     \def\midbox#1{\raisebox{-0.5\baselineskip}[0in][0in]{#1}}
\vspace{3cm}
\title{Assignment 1}
\author{M Pavan Manesh - EE20MTECH14017}
\maketitle
\bigskip
\renewcommand{\thefigure}{\theenumi}
\renewcommand{\thetable}{\theenumi}
%
Download the codes from 
%
\begin{lstlisting}
https://github.com/pavanmanesh/EE5803/tree/main/assign1
\end{lstlisting}
\section{\textbf{Problem}}
Verify the following using Boolean Laws.
\begin{align}
    A^{'}+B^{'}.C=A^{'}.B^{'}.C^{'}+A^{'}.B.C^{'}+A^{'}.B.C+A^{'}.B^{'}.C+A.B^{'}.C \nonumber
\end{align}
\section{\textbf{Boolean Laws Used}}
\begin{itemize}
    \item Complement Law: 
    \begin{align}
        X+X^{'}=1\label{1}
    \end{align}
    \item Distributive Law:
    \begin{align}
        X+Y.Z=(X+Y).(X+Z)\label{2}
    \end{align}
\end{itemize}
\section{\textbf{Solution}}
Consider the right hand side of the given problem:
\begin{align}
 \text{RHS}
 &=A^{'}.B^{'}.C^{'}+A^{'}.B.C^{'}+A^{'}.B.C+A^{'}.B^{'}.C+A.B^{'}.C \\
 &=A^{'}.B^{'}.C^{'}+A^{'}.B(C^{'}+C)+A^{'}.B^{'}.C+A.B^{'}.C \\
 &=A^{'}.B^{'}.C^{'}+A^{'}.B+A^{'}.B^{'}.C+A.B^{'}.C\hspace{0.2cm}\{\text{Using \eqref{1}}\}\\
 &=A^{'}.B+A^{'}.B^{'}(C^{'}+C)+A.B^{'}.C\\
 &=A^{'}.B+A^{'}.B^{'}+A.B^{'}.C\hspace{0.2cm}\{\text{Using \eqref{1}}\}\\
 &=A^{'}(B+B^{'})+A.B^{'}.C\\
 &=A^{'}+A.B^{'}.C\hspace{0.2cm}\{\text{Using \eqref{1}}\}\\
 &=(A^{'}+A)(A^{'}+B^{'}.C)\hspace{0.2cm}\{\text{Using \eqref{2}}\}\\
 &=A^{'}+B^{'}.C\hspace{0.2cm}\{\text{Using \eqref{1}}\}\\
 &=\text{LHS}\nonumber
\end{align}
\section{\textbf{Truth Table}}
\begin{table}[h]
    \centering
    \begin{tabular}{|c|c|c|c|c|}
    \hline
    $A$&$B$&$C$&LHS&RHS  \\
    \hline
    0&0&0&1&1\\
    0&0&1&1&1\\
    0&1&0&1&1\\
    0&1&1&1&1\\
    1&0&0&0&0\\
    1&0&1&1&1\\
    1&1&0&0&0\\
    1&1&1&0&0\\
    \hline
    \end{tabular}
    \caption{Truth Table}
\end{table}
\end{document}
